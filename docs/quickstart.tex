\documentclass[11pt]{article}
\title{\textbf{Chisualizer Quick Start \& Reference}}
\author{Ducky}
\date{24 Nov 2014}
\begin{document}

\maketitle

\section{Prerequisites}
The following software needs to be installed:
\begin{itemize}
  \item Python 2.7 (https://www.python.org/download/releases/2.7/)
  \item wxPython (http://wxpython.org/download.php)
  \item pyCairo (http://cairographics.org/pycairo/)
\end{itemize}

On a Debian or Ubuntu Linux machine, this command should install all the prereqs:
\begin{verbatim}
sudo apt-get python-wxgtk2.8 python-cairo
\end{verbatim}

On a Windows machine, you should be able to download the prereqs from the relevant sites. Note that the examples are reliant on Make and g++, which are not available standard on Windows machines.

\section{Quick Start Example: GCD}
Clone the Chisualizer repository to your location of choice:
\begin{verbatim}
git clone https://github.com/ducky64/chisualizer.git
\end{verbatim}

Go into the GCD example directory, and build the GCD emulator:
\begin{verbatim}
cd chisualizer/tests/gcd
make
\end{verbatim}

Then run the GCD example:
\begin{verbatim}
make run
\end{verbatim}

which is actually shorthand for:
\begin{verbatim}
python ../../src/main.py  --emulator emulator/emulator --visualizer_desc gcd.yaml
pkill emulator
\end{verbatim}

(the pkill is necessary because of a bug where the spawned emulator subprocess does not stop when Chisualizer is terminated, and instead proceeds to eat up all your CPU)

\section{Visualizer Descriptor Language Reference}

\end{document}
